\documentclass{scrartcl}

\usepackage[hidelinks]{hyperref}
\usepackage[none]{hyphenat}
\usepackage{setspace}
\usepackage{graphicx}
\doublespace

\usepackage{amsmath}

\title{Professional Development}

\subtitle{COMP150 - CPD}

\author{1608557}

\begin{document}

\maketitle
\section{Introduction}
This paper covers the aspects of the key skills I feel i need to develop to progress into a more developed student. These key skills are all important and have been implemented into situations relevant in this semester. The \textit{SMART} objectives are set throughout the paper for each key skill as a reflection to becoming a more competent student. Following these target will lead myself to be a more professionally developed student with a mindset of how to face challenges in this type of work.

\section{Key Skills}
In the following sections I aim to cover key skills that I have been challenged on and how I will develop them. 

\subsection{Competency in Learning}
During some of the earlier classes of the year I been introduced to many new programs and features I have never used before. I have found this challenging in the fact I have to catch up in multiple areas but with a little concentration and hard work I don’t think I will be too difficult. One that I am not too keen on is PyGame, however this is probably to lack of use.

To become more competent, I think I should complete more external work and projects that are more personal to learn at least the basics. This will both be fun while learning. I can also learn from my lecturers, other students and internet sources if I need help. I have already found the PyGame site offers huge help with example on how to use each function. Therefore, catching up here shouldn’t be too difficult if I am willing to put in a bit of extra work outside the classroom. Setting myself the goal of completing a personal project in software I am not comfortable with is a great way to help myself become more used to using it when it comes into assignments. Measuring this will be easy and whether it has made any difference can be easily shown too by the quality of the work produced.

\subsection{Preparation}

Linking to the previous point attempting to understand work before lessons I feel is beneficial too. As I myself am new to programming lessons may skip over basic knowledge I do not know causing confusion. An example of this would be algorithm complexities during Ed Powley's lessons have been the most challenging as i have no previous knowledge of anything relevant to this. A prime example for this was Worksheet C that we were set as a task to do outside of class. I felt like this was very hard but now I understand that it is not as hard as i thought. At first i tried to tackle these equations on my own which did not go too well, but managing to complete some of it on my own. At this point I discussed it with some of my peers which helped a small amount. Having now gone through the answer in lecture and explained how each works it is a lot clearer. 

I think in the future I should definitely go over the session slides out of lesson as this seems to be the most useful resource, however I did not make use of it. I am planning to do this in the future as they are easy to access and especially use for worksheet assignments if confused again.

\subsection{Communication in Agile Group Projects}
Group sizes of twelve people is a size I have not worked with before for an assignment. This made it difficult for myself to have sufficient knowledge of how to tackle such work. The smaller pair programming projects were a lot easier to control in comparison. This is something that I believe you cannot practice by yourself. However I believe the best way to conquer my lack of knowledge is to try and form a group of people to complete a personal project during free time. The \textit{SMART} target here would be to produce a certain piece of work meaning the target is measurable. Setting deadlines to manage the time aspect of the project too as a person would in a real group agile project. 

\subsection{Motivation for Assessed Deadlines}
Nearing deadlines dates can be stressful among many assignments and presentations in progress all at once. Currently I have six upcoming hand in dates for various work covering all three modules of the course. Although I have much work to do I have found it hard to motivate myself to get work done and have therefore set myself a target over the coming weeks. I want to set myself smaller deadlines for parts of each assignment. This will lead to me progressively getting work completed rather than doing considerable amount of work near deadlines. This is especially important due to the deadlines being so close together.

\subsection{Routinely and Self Practice}
A challenge during the very first week was understanding basic code and reviewing outcomes when using different integers as the input. One action that helped a lot was going through as a class, step by step, how the code would be executed. Having done some learning on python externally out of lectures I have become a lot more competent. This has allowed me to progress not only reading and understanding blocks of code but actually writing simple programs myself using a varied amount of functions. I feel as if I could have had a better understanding of following lectures by extending my knowledge even before I started the course itself.

Following this I have set myself the goal of continuing to learn python through the use of online tutorials and actual self-practice on a regular basis until I transfer to C++ over the Christmas period. I can measure this goal through the amount of time I have spent learning Python and by how many tutorials I have gone through.

\section{Conclusion}

To conclude this paper, I believe this paper of reflection has helped me evaluate my learning and particularly set out what I personally need to accomplish in order for me to progress. Following these \textit{SMART} targets I have appointed, my performance should increase in the mentioned forms.
This analysis in the five key skills that I personally need further competency has benefited my knowledge for how I will go about certain situations in due course. Having now looked at individual and grouped work with the given targets this has given myself confidence for what is to come in the second semester.



\bibliographystyle{ieeetran}
\bibliography{references}



\end{document}
