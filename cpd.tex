\documentclass{scrartcl}

\usepackage[hidelinks]{hyperref}
\usepackage[none]{hyphenat}
\usepackage{setspace}
\usepackage{graphicx}
\doublespace

\usepackage{amsmath}

\title{Professional Development}

\subtitle{COMP150 - CPD}

\author{1608557}

\begin{document}

\maketitle
\section{Introduction}
This paper covers the aspects of the key skills I feel i need to develop to progress into a more developed student. These key skills are all important and have been implemented into situations relevant in this semester. The \textit{SMART} objectives are set throughout the paper for each key skill as a reflection to becoming a more competent student. Following these target will lead myself to be a more professionally developed student with a mindset of how to face challenges in this type of work.

\section{Key Skills}
In the following sections I aim to cover key skills that I have been challenged on and how I will develop them. 

\subsection{Competency in Learning}
-Competency in my learning
going over previous work and whats to come

\subsection{Preparation}

\subsection{Communication in Agile Group Projects}
-Communication in agile group projects/assignments

\subsection{Motivation for Assessed Deadlines}
-Motivation, toward deadlines and assignments 
using free time links to 

\subsection{Routinely practice}
-Regular practice and preparation 



\section{Conclusion}

To conclude this paper, I believe this paper of reflection has helped me evaluate my learning and particularly set out what I personally need to accomplish in order for me to progress. Following these \textit{SMART} targets I have appointed, my performance should increase in the mentioned forms.
This analysis in the five key skills that I personally need further competency has benefited my knowledge for how I will go about certain situations in due course. Having now looked at individual and grouped work with the given targets this has given myself confidence for what is to come in the second semester.



\bibliographystyle{ieeetran}
\bibliography{references}



\end{document}
